\documentclass{article}



\usepackage{amsmath}

\setlength\parindent{0pt}

\begin{document}

\section{Power Spectrum}
We divide the power spectrum this way:

\begin{equation}
    P(k,a) = P_{\text{ad}}(k,a) + P_{\text{iso}}(k,a)
\end{equation}
where:
\begin{itemize}
    \item 
    \begin{equation}
        P_{\text{ad}}(k,a) = 2\pi^2 A_s \frac{k^{n_s}c^{n_s+3}}{H_0^{n_s+3}} T^2(k) \left(\frac{D(a)}{D(a=1)}\right)^2
    \end{equation}
   The growth factor in a universe with matter and dark energy is:
    \begin{equation}
        \begin{split}
             & D(a) = \frac{5 \Omega_m}{2} \frac{H(a)}{H_0} \int_0^a \frac{da'}{(a' H(a)/H_0)^3}\\
             & H(a) = H_0 \left(\frac{\Omega_m}{a^3}+\Omega_\Lambda \right)^{1/2}
        \end{split}
    \end{equation} 
    By defining:
   \begin{equation}
   \begin{split}
    &\Omega_m(z)=\frac{\Omega_{m,0}}{\Omega_{\Lambda,0}a^3+\Omega_{m,0}}\\
    &\Omega_\Lambda(z)=\frac{\Omega_{\Lambda,0}}{\Omega_{\Lambda,0}+\Omega_{m,0}/a^3}\\
   \end{split} 
   \end{equation}
   we get the following approximation for the growth factor:
   \begin{equation}
    D(z)=\frac{5\Omega_m(z)}{2}\left(\Omega_m(z)^{-4/7}-\Omega_\Lambda(z)+\left(1+\frac{\Omega_m(z)}{2}\right)\left(1+\frac{\Omega_\Lambda(z)}{70}\right)\right)^{-1}
   \end{equation}
   The transfer function is well fitted by:
    \begin{equation}
             T(k) = \frac{\log (1+2.34q)}{2.34q}(1+3.89q+(16.1q)^2+(5.46q)^3+(6.71q)^4)^{-1/4}
    \end{equation} 
    where:
    \begin{equation}
        q \equiv \frac{1}{\Gamma}\left(\frac{k}{h \ Mpc^{-1}}\right) \quad \text{and} \quad \Gamma=\Omega_{m,0}h\exp(-\Omega_{b,0}(1+\sqrt{2h}/\Omega_{m,0}))
    \end{equation}
    \item 
    \begin{equation}
        P_{\text{iso}}(k,a) = 
        \begin{cases}
            \frac{(\text{f}_{\text{PBH}}\tilde{D}(a))^2}{\bar{n}_{\text{PBH}}} & k < k_{\text{PBH}} \\
            0 & k > k_{\text{PBH}} 
        \end{cases}
    \end{equation}
    where $k_{\text{PBH}} = (2\pi^2 \bar{n}_{\text{PBH}}/\text{f}_{\text{PBH}})^{1/3}$, $\bar{n}_{\text{PBH}}=\text{f}_{\text{PBH}}\frac{3H_0^2}{8\pi G}(\Omega_m-\Omega_b)/\text{m}_{\text{PBH}}$ and:
    \begin{equation}
        \begin{split}
            & \tilde{D}(a) = \left(1+\frac{3\gamma}{2a_-}s\right)^{a_-} - 1 \qquad s=\frac{a}{a_{\text{eq}}}\\
            & \gamma = \frac{\Omega_m - \Omega_b}{\Omega_m} \qquad a_- = \frac{1}{4}(\sqrt{1+24\gamma}-1)
        \end{split}
    \end{equation}
    \end{itemize}

\newpage
\section{Halo Mass Function}

The smoothed density variance is:
\begin{equation}
    \sigma(R)^2 = \int_0^\infty \frac{dk}{2\pi^2} k^2 \ P(k) W(k,R)^2
\end{equation}
The window function is taken to be gaussian:
\begin{equation}
    W(x,R) = \frac{1}{(2\pi)^{3/2}R^3}\text{exp}(-x^2/2R^2)
\end{equation}
with a Fourier transform:
\begin{equation}
    W(k,R) = \text{exp}(-k^2R^2/2)
\end{equation}
 The relationship between the radius $R$ and the mass of the halo is:
\begin{equation}
    M(R) = (2\pi)^{3/2}R^3 \bar{\rho} \qquad \bar{\rho} = \rho_{\text{crit},0}\cdot\Omega_m
\end{equation}
therefore:
\begin{equation}
    W(k,M)^2 = \text{exp}\left(-\frac{k^2}{2\pi}\left(\frac{M}{\bar{\rho}}\right)^{2/3}\right)
\end{equation}
Finally the halo mass function is, considering ellipsoidal dynamics with Sheth-Tormen:
\begin{equation}
    \frac{dn}{dM} = -A \sqrt{\frac{2}{\pi}}\sqrt{a} \ \nu \ \text{exp}(-a\nu^2/2) (1+(a\nu^2)^{-p})\frac{\bar{\rho}}{M^2}\frac{\text{d}\log(\sigma(M))}{\text{d}\log(M)}
\end{equation}
with $\nu = \frac{\delta_c}{\sigma(M)}$ and:
\begin{equation}
    A = 0.32 \qquad a=0.75 \qquad p=0.3
\end{equation}.We can rewrite it as:
\begin{equation}
    \begin{split}
     \frac{dn}{dM} &= -\sqrt{\frac{2}{\pi}}\frac{\delta_c}{\sigma} \text{exp}\left(-\frac{a\delta_c^2}{2\sigma^2}\right)(1+(a\nu^2)^{-p})\frac{\bar{\rho}}{M^2}\frac{\text{d}\log(\sigma(M))}{\text{d}\log(M)}\\
                   &= -\sqrt{\frac{2}{\pi}}\frac{\delta_c}{\sigma} \text{exp}\left(-\frac{a\delta_c^2}{2\sigma^2}\right)(1+(a\nu^2)^{-p})\frac{\bar{\rho}}{M^2}\frac{M}{\sigma}\frac{\text{d}\sigma(M)}{\text{d}M}\\
                   &= - \sqrt{\frac{2}{\pi}}\frac{\delta_c}{\sigma^2} \text{exp}\left(-\frac{a\delta_c^2}{2\sigma^2}\right)(1+(a\nu^2)^{-p})\frac{\bar{\rho}}{M}\frac{\text{d}\sigma(M)}{\text{d}M}\\
                   &= - \sqrt{\frac{2}{\pi}}\frac{\delta_c}{\sigma^2} \text{exp}\left(-\frac{a\delta_c^2}{2\sigma^2}\right)(1+(a\nu^2)^{-p})\frac{\bar{\rho}}{M}\frac{1}{2\sigma}\frac{\text{d}(\sigma(M)^2)}{\text{d}M}\\
                   &= - \sqrt{\frac{2}{\pi}}\frac{\delta_c}{\sigma^3} \text{exp}\left(-\frac{a\delta_c^2}{2\sigma^2}\right)(1+(a\nu^2)^{-p})\frac{\bar{\rho}}{M}\frac{1}{2}\frac{\text{d}(\sigma(M)^2)}{\text{d}M}\\
    \end{split}
\end{equation}
Now, the only dependence on $M$ contained in $\sigma(M)^2$ is in $W(k,M)^2$, which is given by (12). Then:
\begin{equation}
    \frac{\text{d}(W(k,M)^2)}{\text{d}M} = -\frac{k^2}{2\pi}\frac{2}{3}\frac{1}{\bar{\rho}^{2/3}M^{1/3}}W(k,M)^2
\end{equation}
Therefore the derivative of $\sigma(M)^2$ is:
\begin{equation}
    \frac{\text{d}(\sigma(M)^2)}{\text{d}M} = -\frac{1}{3\cdot 2\pi}\frac{1}{\bar{\rho}^{2/3}}\frac{1}{M^{1/3}}\int_0^\infty \frac{dk}{\pi^2} k^4 \ P(k) W(k,M)^2
\end{equation}
Putting (16) in (14) we get:
\begin{equation}
    \frac{dn}{dM}=\frac{1}{3\cdot (2\pi)^{3/2}}\frac{\delta_c}{\sigma^3}\frac{\bar{\rho}^{1/3}}{M^{4/3}}\text{exp}\left(-\frac{\delta_c^2}{2\sigma^2}\right) \ (1+(a\nu^2)^{-p})\int_0^\infty \frac{dk}{\pi^2} k^4 \ P(k) W(k,M)^2
\end{equation}
\newpage
\section{Cosmological Parameters}

\begin{center}
\begin{tabular}{ |c|c| } 
  \hline
   $h$ & 0.674 \\ 
  \hline
   $H_0$ & 100 $h$ \ $km  \ s^{-1} \ Mpc^{-1}$\\ 
  \hline
   $\rho_{\text{crit},0}$ & 1.260 $\times \ 10^{11} \ M_\odot \ Mpc^{-3}$   \\ 
  \hline
  $\Omega_m$ & 0.315 \\
  \hline
  $\Omega_\Lambda$ & 0.685 \\
  \hline
  $\Omega_b$ & 0.049 \\
  \hline
  $A_s$ & 2.1 $\times 10^{-9}$ \\
  \hline
  $n_s$ & 0.965 \\
  \hline
  $z_{\text{eq}}$ & 3402 \\
  \hline
  $k_{\text{eq}}$ & 0.015 $h \ Mpc^{-1} $ \\
  \hline 
  $G$ & 4.301 $\times 10^{-9} \ km^2 \ Mpc \ M_\odot^{-1} \ s^{-2} $ \\
  \hline
  $\delta_c$ & 1.69 \\
  \hline 
\end{tabular}
\end{center}

\newpage

\section{Results}

\begin{center}
\begin{tabular}{ |c|c|c|c|c|c| } 
  \hline
   $M_{\star}$& $\epsilon$ & $\text{f}_{\text{PBH}}$ & $\text{m}_{\text{PBH}}$ & $\text{f}_{\text{PBH}}\text{m}_{\text{PBH}}$ & $\rho_\star$   \\
  \hline
   $10^{10}$ & $1$ & $10^{-4}$ & $24\cdot 10^{7}$ & $2.4\cdot 10^{4}$ & \\ 
   $10^{10}$ & $1$ & $10^{-5}$ & $24\cdot 10^{8}$ & $2.4\cdot 10^{4}$ & \\
  \hline
   $10^{10.5}$ & $1$ & $10^{-4}$ & $18\cdot 10^{8}$ & $1.8\cdot 10^{5}$ & \\ 
   $10^{10.5}$ & $1$ & $10^{-5}$ & $18\cdot 10^{9}$ & $1.8\cdot 10^{5}$ & \\
 \hline
   $10^{10}$ & $0.1$ & $10^{-5}$ & $18\cdot 10^{11}$ & $1.8\cdot 10^{7}$ & \\ 
   $10^{10}$ & $0.1$ & $10^{-5}$ & $18\cdot 10^{10}$ & $1.8\cdot 10^{6}$ & \\
\hline
   $10^{10.5}$ & $0.1$ & $10^{-3}$ & $61\cdot 10^{8}$ & $6.1\cdot 10^{6}$ & \\ 
   $10^{10.5}$ & $0.1$ & $3\cdot 10^{-5}$ & $61\cdot 10^{10}$ & $6.1\cdot 10^{6}$ & \\
\hline
\end{tabular}
\end{center}





\end{document}